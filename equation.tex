\section{Equation}

\subsection{Aligning equations}

If you are going to list more than one equation in an aligned manner, you can use the \emph{split} or \emph{aligned} in the package \emph{amsmath}. Two samples are shown as below:

\begin{lstlisting}
\begin{equation} \label{eq:align_sample_1}
a = \sum_{i=1}^{n} i \\
b = \frac{(n+1)(n+2)}{n!}
\end{equation}
\end{lstlisting}

\begin{equation} \label{eq:align_sample_1}
\begin{split}
a &= \sum_{i=1}^{n} i \\
b &= \frac{(n+1)(n+2)}{n!}
\end{split}
\end{equation}

\begin{lstlisting}
\begin{equation} \label{eq:align_sample_2}
\begin{aligned}
F ={} & \{F_{x} \in  F_{c} : (|S| > |C|) \\
      & \cap (\mathrm{minPixels}  < |S| < \mathrm{maxPixels}) \\
      & \cap (|S_{\mathrm{conected}}| > |S| - \epsilon)\}
\end{aligned}
\end{equation}
\end{lstlisting}


\begin{equation} \label{eq:align_sample_2}
\begin{aligned}
F ={} & \{F_{x} \in  F_{c} : (|S| > |C|) \\
      & \cap (\mathrm{minPixels}  < |S| < \mathrm{maxPixels}) \\
      & \cap (|S_{\mathrm{conected}}| > |S| - \epsilon)\}
\end{aligned}
\end{equation}